\let\negmedspace\undefined
\let\negthickspace\undefined
\documentclass[journal,12pt,twocolumn]{IEEEtran}
\usepackage{cite}
\usepackage{amsmath,amssymb,amsfonts,amsthm}
\usepackage{algorithmic}
\usepackage{graphicx}
\usepackage{textcomp}
\usepackage{xcolor}
\usepackage{txfonts}
\usepackage{listings}
\usepackage{enumitem}
\usepackage{mathtools}
\usepackage{gensymb}
\usepackage{comment}
\usepackage[breaklinks=true]{hyperref}
\usepackage{tkz-euclide}
\usepackage{listings}
\usepackage{gvv}
\def\inputGnumericTable{}
\usepackage[latin1]{inputenc}
\usepackage{color}
\usepackage{array}
\usepackage{longtable}
\usepackage{calc}
\usepackage{multirow}
\usepackage{hhline}
\usepackage{ifthen}
\usepackage{lscape}

\newtheorem{theorem}{Theorem}[section]
\newtheorem{problem}{Problem}
\newtheorem{proposition}{Proposition}[section]
\newtheorem{lemma}{Lemma}[section]
\newtheorem{corollary}[theorem]{Corollary}
\newtheorem{example}{Example}[section]
\newtheorem{definition}[problem]{Definition}
\newcommand{\BEQA}{\begin{eqnarray}}
\newcommand{\EEQA}{\end{eqnarray}}
\newcommand{\define}{\stackrel{\triangle}{=}}
\theoremstyle{remark}
\newtheorem{rem}{Remark}
\begin{document}

\bibliographystyle{IEEEtran}
\vspace{3cm}

\title{Gate 2023- Instrumentation Engineering}
\author{EE23BTECH11058 - Sindam Ananya$^{*}$% <-this % stops a space
}
\maketitle
\newpage
\bigskip

\renewcommand{\thefigure}{\theenumi}
\renewcommand{\thetable}{\theenumi}

\vspace{3cm}
\textbf{Question 60:} 
In the circuit shown, the input voltage $V_{in} = 100mV$. The switch and the opamp are ideal. At time $t=0$, the intial charge stored in the $10nF$ capacitor is $1nC$, with the polarity as indicated in the figure. The switch $S$ is controlled using a $1KHz$ square-wave voltage signal $V_s$ as shown. Whenever $V_s$ is 'High', $S$ is in position $'1'$ and when $V_s$ is 'Low', $S$ is in position $'2'$.\\
At $t = 20ms$, the magnitude of the voltage $V_o$ will be  \\  
\solution
\begin{table}[h!]
    \centering
    \begin{tabular}{|c|c|c|}
        \hline
        \textbf{Parameter} & \textbf{Value} & \textbf{Description} \\
        \hline
        $V_{in}$ & $100mV$ & Input voltage \\
        \hline
        $q_{10nF}$ & $1nC$ & Intial charge on $10nF$ \\
        \hline
        $q_{1nF}$ & & Charge on $1nF$\\
        \hline
        $f$ & $1KHz$ & Frequency of $V_s$ \\
        \hline
        $T$ & $1ms$ & Time period\\
        \hline
        $V_o$ &  & Output voltage \\
        \hline
    \end{tabular}

    \caption{Input Parameters}
    \label{tab:gatein60table}
\end{table}
As $f = 1KHz$ its time-period is $1ms$.\\
So, for $0.5ms$ the switch will be at position $1$ and then shift to position $2$ for the other $0.5ms$ and this continues in cycles.\\ \\
In the first $0.5ms$ the switch will be at position $1$. The $1nF$ capacitor gets charged to $0.1nC$.\\ \\
In the second $0.5ms$ the switch shift to position $2$. So, the op-amp's $-ve$ terminal gets virtually shorted then the charge across the $1nF$ capacitor should be zero.\\ \\
To maintain the charge across the $1nF$ to be $0$, it transfers the charge of $0.1nC$ to the $10nF$ capacitor.\\ \\
As the polarity across the $10nF$ capacitor is opposite to the which the $1nF$ capacitor transfers the charge the charge gets subtracted.\\ \\
So, effectively in a cycle of $1ms$ time-period $-0.1nC$ charge is transfered to the $10nF$ capacitor. In $20$ cycles it transfers $-2nC$.\\ \\
At $t=20ms$ the effective charge on the $10nF$ capacitor is $-1nC$ and voltage across it is $V_o$.\\
\begin{align}
V_o &= \frac{-1nC}{10nF}\\
    &= -100mV
\end{align}
\end{document}
